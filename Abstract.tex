\chapter*{ABSTRACT}
\thispagestyle{fancy}
\addcontentsline{toc}{chapter}{ABSTRACT}

\begin{center}
  \TITLE \\[36pt]
  By \\ [12pt] \AUTHOR \\
  \ADVISOR, Advisor \\ \COADVISOR, Co-Advisor \\[0.3in]
  \text{SWISS GERMAN UNIVERSITY} \\[36pt]
  
\end{center}
Begin typing the abstract here, 1.5 spaced. The abstract must include the following components: purpose of the research, methodology, findings, and conclusion. The abstract must it is imperative to prioritize clarity, conciseness, accuracy, and a logical flow. Begin by succinctly summarizing the key objectives, methods, and findings of your research. Avoid jargon and overly technical language, opting instead for clear and accessible terminology. Ensure that your abstract accurately reflects the content of your thesis, avoiding any misleading statements. Review and revise your abstract meticulously, eliminating unnecessary details and maintaining a concise yet informative tone throughout. Lastly, In the abstract of your thesis, ensure utmost clarity in conveying the research's objectives and its overall significance. Begin by succinctly articulating the main goals of your study, outlining what you aimed to investigate, discover, or achieve. Subsequently, expound on the relevance and broader implications of your research in the respective field or context. Highlight how your findings contribute to existing knowledge, address gaps, or potentially impact practical applications. Keep the abstract concise, focusing on conveying essential information rather than delving into intricate details. 

\vfill

\noindent \emph{Keywords}: Keyword1, Keyword2, Keyword3, Keyword4, Keyword5 (use scientific terms).
