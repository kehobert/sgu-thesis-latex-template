
\chapter*{Glossary}
\label{chap:glosarry}
\thispagestyle{fancy}
\addcontentsline{toc}{chapter}{GLOSSARY}

 \begin{table}[ht!]
	\centering
	\footnotesize
	\renewcommand{\arraystretch}{2}
	%\caption{Glossary)}
	\label{tab:glossary}
	\vspace{2mm}
	%\begin{tabular}{@{}clccccc@{}}
	\begin{tabular*}{1.0\textwidth}{@{}p{0.16\linewidth}*{4}{p{0.8\linewidth}}}
		\textbf{BENIGN} & In term of computer, benign application is software that has no malicious effect on computer if it's running. The opposite is malware \\
		
		\textbf{DBI} & DBI is shorthand of Dynamic Binary Instrumentation. It is a system that enables injecting new codes and manipulating binary during runtime without the need of source code. \\
		
		\textbf{DYNAMIC ANALYSIS} & Analysis of binary by running it in a controlled environment. \\
		
		\textbf{EMULATION} & In terms of computing, it refers to  reproduction of system function on different type of computing platform. For example running ARM-based software on x86 host. \\
		
		\textbf{GUEST} & The operating system installed inside a virtual machine. \\
		
		\textbf{HOST} & The computer that runs virtual machine that runs Ubuntu OS that runs PANDA. \\
		
		\textbf{KVM} & Short for Kernel-based Virtual Machine, which turns Linux kernel into a hypervisor in virtualization infrastructure. \\
		
		\textbf{KVM} & Short for Kernel-based Virtual Machine, which turns Linux kernel into a hypervisor in virtualization infrastructure. \\
		
		\textbf{MALWARE} & Short for malicious software. When it runs in a computer system, malware has destructive tendency toward its infected system. The opposite is Benign application. \\
		
		\textbf{ORIGINAL ENTRY POINT} & Also known as OEP, an entry point for original code which already been tampered by means of packer or compressed. \\
		
		\textbf{PE} & Short for Portable Executable. Famously known as binary executable files that are run on Windows environment. \\
		
		\textbf{PEiD} & 3rd party tools that detect Windows Portable Executables whether its packed or not. \\
		
	\end{tabular*}
\end{table}
